%\documentstyle{letter}
\documentclass[
%reprint,
preprint,
onecolumn,
 superscriptaddress,
 amsmath,amssymb,
 aps,
]{revtex4-1}

\usepackage{xcolor}
\linespread{1.1}

\begin{document}

%\begin{letter}{
%Professor Reichmann\\
%Journal of Chemical Physics\\
%Department of Chemistry\\
%5735 S. Ellis St.\\
%University of Chicago\\
%Chicago, IL 60637}
%\address{}
%\signature{
%\parskip 1 cm
%Junhao Li\\
%Physics Department\\
%Cornell University\\
%Clark Hall, Ithaca, NY 14853\\
%}
%\opening{Dr. Reichmann:}

\noindent Manuscript No.: A18.09.0071 \\
Title: Fast Semistochastic Heat-Bath Configuration Interaction \\
Authors: Junhao Li, Matt Otten, Adam Holmes, Sandeep Sharma, and Cyrus Umrigar
\vskip 8mm

\noindent Dear Professor Reichman, \\
Thank you for sending us the referee reports.  We thank the referees for taking the time to
read the manuscript
and for making suggestions to improve it.

We have responded to each of the referee comments below and made suitable changes to the
manuscript.
We hope that you and the referees will deem the revised manuscript suitable for publication.

\vskip 3mm
\noindent Sincerely,\\
Junhao Li, Matt Otten, Adam Holmes, Sandeep Sharma, and Cyrus Umrigar

\vskip 8mm

\noindent Here are the responses to each of the reviewer comments:

\noindent\underline{Reply to reviewer 1}

\vskip 5mm {\color{blue}
This manuscript reports a new, more efficient implementation of the author's semi-stochastic
heat-bath configuration interaction (SHCI). Selected CI is currently a "hot topic" in quantum
chemistry, and this work presents some exciting technical developments that allow the authors
to treat over 1 billion determinants variationally and estimate perturbative corrections for
trillions of determinants. To handle such unprecedented large determinant spaces, the authors
improved the speed of the SHCI algorithm using a series of very clever tricks, including
batching and subdividing the stochastic PT2 energy corrections into three contributions. Some
of the technical improvements included in the fast SHCI are new, and others have been known
to the CI community for some time. The paper is highly technical as it focuses significantly
on the algorithmic details of the fast SHCI method. However, I think that the value of this
work is really in providing the "recipe" to implement a fast SHCI code, which I am sure other
researchers in this field will find useful. The authors have benchmarked their method using
two challenging systems (Cu atom, Cr2). The Cr2 computations are an impressive computational
feat and show that this version of SHCI is competitive with well-established methods like DMRG
and p-DRMG. Again, although it is a bit technical, I think that the paper certainly deserves
publication in JCP. I have a few remarks that the authors should consider before acceptance:
}\color{black}

We are glad that the referee appreciates our "impressive computational feat".  We agree the
paper is rather technical, but in our opinion
this is an important ingredient of the paper -- we not only improve the basic algorithm used,
but we also provide technical details that
are essential for efficient calculations on large systems which would be very useful to other
researchers who wish
to implement the method.

\vskip 5mm {\color{blue}
1. It is nice to see such a good agreement of SHCI with DMRG for Cr2 at equilibrium geometry
(1.68 Ang.), but to test the limits of this approach, the authors should report computations
at larger bond distances, say at 1.5-2 r\_e, where methods like CCSD(T) ultimately fail. There
is no doubt that a DZ basis is insufficient to capture the correct potential for Cr2, but the
issue at stake here is different. The authors should at least comment on whether or not these
computations are possible, or better try to do them and report them in the paper. Showing that
SHCI can compute Cr2 away from equilibrium would give a big boost to the paper's impact. The
authors also fail to give credit to the many researchers that have previously studied Cr2 with
an array of diverse computational methods, e.g., see papers by Roos and co-workers, Scuseria,
Chan, etc.. This issue should be easy to rectify.
}\color{black}

We are indeed working on calculating the entire potential energy curve for Cr$_2$.  Since this
involves a very large computational effort,
we plan to publish this in a separate paper.  As the reviewer implies, the computations at the
stretched geometry are in fact
more expensive than those at equilibrium, but the difference is not huge and we are getting an
accurate and smooth curve.
We did not cite other papers on Cr$_2$ because Cr$_2$ is not the point of this paper -- it is
merely used as an example to demonstrate
the efficiency of the method.  Nevertheless, we have added x references as suggested by the
referee.

\vskip 5mm {\color{blue}
2. With the new implementation, the authors introduce additional energetic cutoff parameters
used in the perturbative correction separated into three parts (deterministic, pseudo-stochastic,
and stochastic). I would say that this is an undesirable feature, but several recent methods have
started to use several parameters that determine the truncation of small quantities (e.g. Neese's
DLPNO methods). This new scheme is a significant focus of the publication, but I think that
a discussion of the sensitivity of the energy with respect to these parameters (perhaps via a
comparative illustration or figure) and how those parameters are chosen relative to one another
for a given type of system would be good to include. Currently, the paper only says that these
'depend on the system' and give 'reasonable' values for them, but I think elaborating on the
rationale behind how these parameters are chosen would be very helpful.
}\color{black}

It is true that the 3-step perturbation theory (PT) has more parameters than our previous 2-step
PT, however this is well worthwhile
because of the large gain in efficiency.  It should be noted that the accuracy of the PT does
NOT depend on the additional parameters.
It is only the efficiency that depends on them.  The accuracy depends only on $\epsilon_2$
and the increased efficiency of the 3-step PT
allows us to use a really tiny value of $\epsilon_2$ which guarantees highly converged results.
We have added a paragraph explaining
this on pg. xx.

\vskip 5mm {\color{blue}
3. Perhaps this is covered in the earlier heat bath CI papers, but it might be a good idea to
discuss the statistical error associated with stochastic sampling. The improvements in this
regard are mentioned throughout the article, but it may be helpful to the reader to include a
paragraph in the introduction (or the HCI review section) about stochastic vs. deterministic
sampling, and what is it known about how the former introduces statistical errors.
}\color{black}

%We are not completely sure what the referee is asking here, so we make a couple of points that
%we hope will clarify matters.
The referee asks about "stochastic vs. deterministic sampling" but of course it does not make
sense to talk about "deterministic sampling", so we assume he is asking us why we choose to
use stochastic sampling.  This is already explained on pg. 8, where we compare to two other
recent algorithms.  Basically, if one were to do a deterministic PT calculation with a very
large number of variational determinants, we would run out of both computer memory and computer time.
We are able to use two orders of magnitude more variational determinants than the largest
calculation by any other method because we use stochastic sampling.
Another point to make is that usually the statistical error goes down as $1/\sqrt{N_{MC}}$, i.e.
as $1/\sqrt{T}$ where $N_{MC}$ is the number of Monte Carlo samples and $T$ is the computer time.
However, as discussed in the caption of Table 2, in our approach the error goes down much faster because
we have a multi-step PT method and we do a larger deterministic and pseudostochastic calcultion,
rather than merely increasing the number of samples in the stochastic step.
To further clarify matters, we have added in a paragraph on pg. xx.


\vskip 5mm {\color{blue}

4. In the past couple of years, several other groups have been working on or have used HCI and
other CIs to compute strongly correlated systems. The authors seem to be oblivious of other
works by Ten-no, Zimmerman, Greer, Coe and Patterson, Evangelista, Scuseria, Gagliardi. I
suspect some of these might be very relevant to this paper. The authors should also cite the
ASCI work by Head-Gordon and Whaley that published recently in JCP.
}\color{black}

We have already cited a selected CI plus perturbation theory (SCI+PT) paper by Evangelista.
The other papers as far as we know are not
closely related to our work, but we have nevertheless added citations to papers by ...

\vskip 5mm {\color{blue}

5. The discussion of hash tables mentions that a '... distributed hash table is based on
lock-free open- addressing linear-probing concurrent hash tables specifically designed for
intensive commutative insertion and update operations.' I would assume that most readers of
JCP are not familiar with the CS technical jargon used here. It would be advisable to add a
reference and perhaps explain a bit more what these terms mean.
}\color{black}

We agree that most JCP readers are not familiar with these CS techniques and we have added in
a citation.

%------------------------------------------------------------------------------
\vskip 8mm
\noindent\underline{Reply to reviewer 2}

\vskip 5mm {\color{blue}

Reviewer \#2 (Comments to the Author):

The authors provide a detailed description of a number of SHCI algorithm improvements that
allow them to work with larger variational CI expansions than were previously possible in the
theory. These improvements assist with both the variational and perturbative stages of the
calculation, and ultimately allow for an impressive demonstration on the Cr dimer in an active
space that only p-DMRG was capable of reaching before. Their results both confirm and perhaps
slightly improve on the p-DMRG numbers. This paper will be suitable for publication in JCP,
although it requires significant revisions, mostly for clarity, before publication.
}\color{black}

Thanks for the positive comments.

\vskip 5mm {\color{blue}

1. The authors make various strong statements comparing DMRG and FCIQMC in their
introductions. Citations should be provided to substantiate these statements.
}\color{black}

There have been no published studies that compare the relative efficiencies of these methods,
but the authors of this paper have
between us contributed to all 3 methods.  Some of us are planning to publish a separate paper
that discusses the strengths and weaknesses
of the various methods.

\vskip 5mm {\color{blue}
2. Why do the authors not mention the related ASCI method in the introduction?
}\color{black}

We do mention and discuss the ASCI paper arXiv 1808.02049v1 since it is relevant to our work.
We do not mention the 2016 ASCI J. Chem. Phys. paper because the minor innovation (excite from
only the more important determinants) to the CIPSI paper that they
describe in their paper had already been used in the 1983 paper of Evangelisti, Daudey and Malrieu.
We also do not mention the arXiv 1807.00821v1 because in our opinion it has many incorrect or
misleading statements.  Contrary to what is stated in that paper, SHCI is more related to CIPSI
than to ASCI, because in SHCI all current determinant can generate new determinants.

\vskip 5mm {\color{blue}
3. Below equation 10, do the authors mean that DeltaE\_2\^s[epsilon\_2] and
DeltaE\_2\^s[epsilon\_2\^d] are computed on the same sample? They appear to have left of the
"s" superscript here.
}\color{black}

We thank the referee for catching these typos and we have corrected them.

\vskip 5mm {\color{blue}
4. The author's description of why a large number of small samples has a larger error than
a small number of large samples is unclear. A more accessible explanation will be especially
helpful to those quantum chemists less familiar with stochastic methods.
}\color{black}

\vskip 5mm {\color{blue}

5. Why does the step 1 batching provide a solution to the memory problem? For that matter,
what object is it that is causing them to run out of memory in practice? Readers will better
follow why the 3-step with batching is helpful if they are shown what specifically is causing
the memory problems in the first place.
}\color{black}

\vskip 5mm {\color{blue}

6. As batching seems to be the key here, a clearer explanation of how the batches are chosen
and why it helps with memory (and statistics, see below) seems merited.
}\color{black}

\vskip 5mm {\color{blue}

7. The main speedup here appears to be that their new 3-step PT2 method is attempting to avoid
evaluating corrections from whole batches of determinants by assuming that the statistics
they establish on a small number of batches will carry over to all of the other (unevaluated)
batches. This sounds like a strong assumption, and the small discussion they provide about
this is not entirely convincing, partly because it is hard to follow, with the logic scattered
over two pages of text. Why are they so confident the different batches' contributions will
be distributed in the same way such that this assumption is accurate? This probably goes back,
at least in part, to the question of how the batches are chosen.
}\color{black}

\vskip 5mm {\color{blue}

8. After repeatedly pointing out the value of the deterministic stage of their original
2-stage and now 3-stage PT theory, the authors skip it in their Cr2 demonstration. This is
quite confusing. Is this stage worth having, or not?
}\color{black}

We agree, this was confusing.  The point is that in this case our $\epsilon_1$ is already
small enough that doing the deterministic PT step would not affect the efficiency much,
so we skip it.  We have added in an explanation on pg. xx.

\end{document}
